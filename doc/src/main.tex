\documentclass[12pt]{article}
\usepackage[utf8]{inputenc}
\usepackage{geometry}
\usepackage{amsmath}
\usepackage{amsfonts}
\usepackage{bm}
\usepackage{hyperref}
\usepackage{cleveref}
\usepackage{tikz}
\usepackage{fancyhdr}
\usepackage{graphicx}

% Page layout
\geometry{a4paper, margin=3cm}

% Paragraph formatting
\setlength{\parindent}{0pt}
\setlength{\parskip}{\baselineskip}

\pagestyle{fancy}
\fancyhf{}
\fancyhead[L]{Project APEX}
\renewcommand{\footrulewidth}{0.4pt}
\fancyfoot[RE,LO]{\thepage}

\begin{document}

\begin{titlepage} % Suppresses displaying the page number on the title page and the subsequent page counts as page 1

	\raggedleft % Right align the title page

	\rule{1pt}{\textheight} % Vertical line
	\hspace{0.05\textwidth} % Whitespace between the vertical line and title page text
	\parbox[b]{0.75\textwidth}{ % Paragraph box for holding the title page text, adjust the width to move the title page left or right on the page

		{\Huge\bfseries Project APEX}\\[2\baselineskip] % Title
		{\large\textit{A Laptime Optimisation Tool}}\\[4\baselineskip] % Subtitle or further description
		{\Large\textsc{Ranbir Singh}} % Author name, lower case for consistent small caps

		\vspace{0.5\textheight} % Whitespace between the title block and the publisher

		{\noindent \today} \\[\baselineskip] % Publisher and logo
	}

\end{titlepage}

\newpage
\tableofcontents

\newpage
\section{Background on Collocation}
Assume some arbitrary dynamic system with state $\bm{x} \in \mathbb{R}^{n_x}$ and external input $\bm{u} \in \mathbb{R}^{n_u}$,
can have its dynamics described by the differential equation
\begin{equation} \label{eq:dynamics}
    \dot{\bm{x}}(t) = f(\bm{x}(t),\, \bm{u}(t),\, t),
\end{equation}
where $t$ is the time and $f: \mathbb{R}^{n_x} \times \mathbb{R}^{n_u} \times \mathbb{R} \to \mathbb{R}^{n_x}$ is a function defining the state derivative $\dot{\bm{x}}(t)$.
A question that will often arise when dealing with such systems is to find a control trajectory $\bm{u}(t)$ that is optimal
with respect to some cost functional $J(\bm{x}(t),\, \bm{u}(t))$ over a time horizon $[t_0,\, t_f]$ given some initial state $\bm{x}(t_0) = \bm{x}_0$, and possibly some additional constaints on the states and inputs.
The cost functional $\Phi$ may have either a Lagrange term, Mayer term, or both, and can generally be expressed as
\begin{equation} \label{eq:cost_functional}
    J(\bm{x}(t),\, \bm{u}(t)) = \underbrace{\int_{t_0}^{t_f} L(\bm{x}(t),\, \bm{u}(t),\, t) \, dt}_{\text{Lagrange term}} + \underbrace{\Phi(\bm{x}(t_f))}_{\text{Mayer term}},
\end{equation}
where $L: \mathbb{R}^{n_x} \times \mathbb{R}^{n_u} \times \mathbb{R} \to \mathbb{R}$ is a cost integrand and $\Phi: \mathbb{R}^{n_x} \to \mathbb{R}$ is some terminal cost.

From this, we can formulate the optimal control problem (OCP)
\begin{subequations} \label{eq:ocp}
\begin{align}
    \underset{\bm{u}(t)}{\text{min}} \quad & J(\bm{x}(t),\, \bm{u}(t)) \label{eq:ocp_cost} \\
    \text{subject to} \quad & \dot{\bm{x}}(t) = f(\bm{x}(t),\, \bm{u}(t),\, t), \quad t \in [t_0,\, t_f], \label{eq:ocp_dynamics} \\
    & \bm{x}(t_0) = \bm{x}_0, \label{eq:ocp_initial} \\
    & \bm{g}_L \leq \bm{g}(\bm{x}(t),\, \bm{u}(t),\, t) \leq \bm{g}_U, \quad t \in [t_0,\, t_f], \label{eq:ocp_constraints}
\end{align}
\end{subequations}
where $\bm{g} : \mathbb{R}^{n_x} \times \mathbb{R}^{n_u} \times \mathbb{R} \to \mathbb{R}^{n_g}$ is a vector of constraint functions with lower and upper bounds $\bm{g}_L,\; \bm{g}_U \in \mathbb{R}^{n_g}$ respectively.

In general, solving OCPs on the form of \eqref{eq:ocp} is not trivial and often requires numerical methods.
One such approach is to transcribe the continuous-time OCP into a finite-dimensional nonlinear programming (NLP) problem using a direct collocation method.
This involves discretising the time horizon $[t_0,\, t_f]$ into $N_{el}$ intervals with boundaries at the points $t_0 = \tau_0 < \tau_1 < \ldots < \tau_N = t_f$.
Section \ref{sec:colloc} describes how the states and inputs are approximated within each interval and Section \ref{sec:nlp_formulation} outlines the formulation of the resulting NLP.

\subsection{Collocation Points} \label{sec:colloc}
Within each interval $[\tau_k,\, \tau_{k+1}]$, the state and input trajectories are approximated as polynomials.
Specifically, Lagrange polynomials are used along with the state values $\bm{x}(\xi_i)$, where $\xi_i \in [\tau_k,\, \tau_{k+1}]$ are the $N_c$ collocation points.
There are several methods for choosing the collocation points, but in this work, we use flipped Legendre-Gauss-Radau (fLGR) points which are described in more detail in Section \ref{sec:flgr_points}.

The polynomial approximation of the state within the interval $[\tau_k,\, \tau_{k+1}]$ is given by
\begin{equation} \label{eq:state_polynomial}
    \bm{x}(t) \approx \sum_{i=0}^{N_c} \bm{x}(\xi_{i}) L_i(\zeta(t)),
\end{equation}
where $\zeta: [\tau_k,\, \tau_{k+1}] \to [-1,\, 1]$ is the mapping defined by
\begin{equation}
    \zeta(t) = \frac{2(t - \tau_k)}{\tau_{k+1} - \tau_k} - 1,
\end{equation}
and the Lagrange basis polynomials are defined by
\begin{equation} \label{eq:lagrange_basis}
    L_i(\zeta) = \prod_{\substack{j=0 \\ j \neq
    i}}^{N_c} \frac{\zeta - \zeta_j}{\zeta_i - \zeta_j}, \quad i = 0,\, \ldots,\, N_c, \quad \zeta \in [-1,\, 1].
\end{equation}

\subsubsection{Flipped Legendre-Gauss-Radau Points} \label{sec:flgr_points}

\subsection{Formulation of the Nonlinear Program} \label{sec:nlp_formulation}

\section{The Models}
\subsection{Point Mass Model}
The Point Mass Model (PMM) is a very simplified representation of a vehicle as a planar point mass.
It's dynamics are described by the differential equations
\begin{equation}
    m\ddot{x} = F_x, \quad
    m\ddot{y} = F_y,
\end{equation}
where $x$ and $y$ are the coordinates of the vehicle in the plane,
$F_x$ and $F_y$ are the forces acting on the vehicle in the $x$- and $y$-directions respectively,
and $m$ is the mass of the vehicle.
\end{document}
